% \usetheme{frankfurt}
\usetheme{Madrid}
% \usetheme{Boadilla}
% \input{Template.tex}
\definecolor{myorange}{RGB}{190,110,0} 
\setbeamercolor*{structure}{fg=myorange}

\usepackage{color}
\usepackage{subcaption}


\usepackage{url}
\usepackage{xspace}
\usepackage{tikz}
%\usepackage{subfigure}
%\usepackage{sub-float}

\usepackage{amsfonts, dsfont}
\usepackage{amsmath}
\usepackage{amssymb}
\usepackage{amsbsy}
\usepackage{amsthm}
\usepackage[mathcal]{euscript}
\usepackage{algorithmic}
\usepackage{movie15}
\usepackage {textpos}
%\usepackage{media9}
%\usepackage{multimedia}
\usepackage{cancel} %for strikethrough
\usepackage{adjustbox}
% \usepackage{physics}
\usepackage[style=authoryear]{biblatex}
%\addbibressource{NewRefs.bib}
\bibliography{NewRefs}

\usepackage{multirow}
\usepackage{colortbl}
\usepackage{bm}


%\usepackage{bibentry}
% \input{Macros_krish}
\usepackage{pdfcomment}
\newcommand{\pdfmovie}[4]{\href{run:#1}{\framebox{\parbox[c][#3][c]{#2}{\center #4}}}}


% tikz stuff
%\usetikzlibrary{mindmap,trees,arrows,shapes,backgrounds,matrix,decorations.pathreplacing,decorations.pathmorphing,positioning}
\usepackage{tikz}
\usetikzlibrary{arrows,shapes}
\usetikzlibrary{decorations.pathreplacing,calligraphy}
\usetikzlibrary{patterns}
\usetikzlibrary{arrows,shapes}
\usetikzlibrary{decorations.pathreplacing,calligraphy}

\usepackage[latin1]{inputenc}

% \addtobeamertemplate{frametitle}{}{
%   \begin{textblock*} {100mm} (0.93\textwidth, -0.5cm)
%      \pgfuseimage {utbig}
%   \end{textblock*}
% }

\usetikzlibrary{calc,3d}
%set the plot display orientation

%% Beamer Stuff
% \useheadtemplate{%
% }
% \usefoottemplate{}
% \userightsidebartemplate{0cm}{%
% }
% \setbeamercolor{background canvas}{bg=}

\newcommand{\R}{\text{\sf I\!\!\:R}}

\newcommand{\mb}[1]{\mathbf{#1}}
\newcommand{\mc}[1]{\mathcal{#1}}
\newcommand{\pp}[2]{\frac{\partial #1}{\partial #2}}
\newcommand{\pps}[3]{\frac{\partial^2 #1}{{\partial #2}{\partial #3}}}
\newcommand{\ppss}[2]{\frac{\partial^2 #1}{{\partial {#2}}^2}}
\newcommand{\LRp}[1]{\left( #1 \right)}
\newcommand{\LRs}[1]{\left[ #1 \right]}
\newcommand{\LRa}[1]{\left< #1 \right>}
\newcommand{\LRc}[1]{\left\{ #1 \right\}}

%% \newcommand{\mygreen}[1]{{\color[rgb]{0,.65,0} #1}}
%% \newcommand{\mywhite}[1]{{\color[rgb]{1.0,1.0,1.0} #1}}
%% \newcommand{\myred}[1]{{\color[rgb]{0.65,0.0,0.0} #1}}
%% \newcommand{\myblue}[1]{{\color[rgb]{0.0,0.0,1.0} #1}}
%% \newcommand{\mygray}[1]{{\color[rgb]{.15,.35,.60} #1}}


% \renewcommand{\frac12}{\tfrac}{1}{2}

\newcommand{\Minimize}{\hspace{0.2cm} \mbox{minimize} \hspace{0.2cm}}
\newcommand{\Maximize}{\hspace{0.2cm} \mbox{maximize} \hspace{0.2cm}}
\newcommand{\SubjectTo}{\hspace{0.2cm} \mbox{subject} \hspace{0.15cm}
            \mbox{to} \hspace{0.2cm}}
\newcommand{\Div}{\nabla \cdot}
\newcommand{\Grad}{{\nabla}}
\newcommand{\Curl}{{\nabla \times}}
\newcommand{\comment}[1]{}
%\newcommand{\m}[1]{{\bf #1}}
%\newcommand{\w}[1]{{\bf #1}}
%\newcommand{\tr}[1]{\mbox{tr}({\bf #1})}

\newcommand{\half}{\frac{1}{2}}




% new theorems
\newtheorem{proposition}{Proposition}

\newcommand{\figlab}[1]{\label{fig:#1}}
\newcommand{\eqnlab}[1]{\label{eq:#1}}
\newcommand{\theolab}[1]{\label{theo:#1}}
\newcommand{\corolab}[1]{\label{coro:#1}}
\newcommand{\propolab}[1]{\label{propo:#1}}
\newcommand{\lemlab}[1]{\label{lem:#1}}
\newcommand{\defilab}[1]{\label{defi:#1}}
\newcommand{\remalab}[1]{\label{rema:#1}}
\newcommand{\tablab}[1]{\label{tab:#1}}

\newcommand{\figref}[1]{\ref{fig:#1}}
\newcommand{\theoref}[1]{\ref{theo:#1}}
\newcommand{\defiref}[1]{\ref{defi:#1}}
\newcommand{\remaref}[1]{\ref{rema:#1}}
\newcommand{\cororef}[1]{\ref{coro:#1}}
\newcommand{\proporef}[1]{\ref{propo:#1}}
\newcommand{\lemref}[1]{\ref{lem:#1}}
\newcommand{\eqnref}[1]{\eqref{eq:#1}}
\newcommand{\alglab}[1]{\label{alg:#1}}
\newcommand{\algref}[1]{\ref{alg:#1}}
\newcommand{\seclab}[1]{\label{sect:#1}}
\newcommand{\secref}[1]{\ref{sect:#1}}
\newcommand{\tabref}[1]{\ref{tab:#1}}



% \setbeamertemplate{background canvas}{%
%  \includegraphics[width=\paperwidth,height=\paperheight]{background_slide}}

%% color theme
%\usecolortheme{fly}
%% \setbeamercolor{normal text}{fg=white,bg=black!90}
%% \setbeamercolor{structure}{fg=white}

%% \setbeamercolor{alerted text}{fg=red!85!black}

%% \setbeamercolor{item projected}{use=item,fg=black,bg=item.fg!35}

%% \setbeamercolor*{palette primary}{use=structure,fg=structure.fg}
%% \setbeamercolor*{palette secondary}{use=structure,fg=structure.fg!95!black}
%% \setbeamercolor*{palette tertiary}{use=structure,fg=structure.fg!90!black}
%% \setbeamercolor*{palette quaternary}{use=structure,fg=structure.fg!95!black,bg=black!80}

%% \setbeamercolor*{framesubtitle}{fg=white}

%% \setbeamercolor*{block title}{parent=structure,bg=black!60}
%% \setbeamercolor*{block body}{fg=black,bg=black!10}
%% \setbeamercolor*{block title alerted}{parent=alerted text,bg=black!15}
%% \setbeamercolor*{block title example}{parent=example text,bg=black!15}
%-------------------------

\usepackage{tikz}

\usetikzlibrary{arrows,shapes}
% For every picture that defines or uses external nodes, you'll have to
% apply the 'remember picture' style. To avoid some typing, we'll apply
% the style to all pictures.
\tikzstyle{every picture}+=[remember picture]

% By default all math in TikZ nodes are set in inline mode. Change this to
% displaystyle so that we don't get small fractions.
\everymath{\displaystyle}
\tikzstyle{na} = [baseline=-.5ex]
\newcommand{\zapspace}{\topsep=0pt\partopsep=0pt\itemsep=0pt\parskip=0pt}

\setbeamerfont{Small}{size=\small}

%%------------Different color of box and theorems-------------------
\definecolor{pacificorange}{cmyk}{.15,.45,1,0} %approved Pacific colors 2010
\definecolor{pacificgray}{cmyk}{0,.15,.35,.60}
\definecolor{pacificlgray}{cmyk}{0,0,.2,.4}
\definecolor{pacificcream}{cmyk}{.05,.05,.15,0}
\definecolor{deepyellow}{cmyk}{0,.17,.80,0}
\definecolor{lightblue}{cmyk}{.49,.01,0,0}
\definecolor{lightbrown}{cmyk}{.09,.15,.34,0}
\definecolor{deepviolet}{cmyk}{.79,1,0,.15}
\definecolor{deeporange}{cmyk}{0,.59,1,18}
\definecolor{dustyred}{cmyk}{0,.7,.45,.4}
\definecolor{grassgreen}{RGB}{92,135,39}
\definecolor{pacificblue}{RGB}{59,110,143}
\definecolor{pacificgreen}{cmyk}{.15,0,.45,.30}
\definecolor{deepblue}{cmyk}{1,.57,0,2}
\definecolor{turquoise}{cmyk}{.43,0,.24,0}
\definecolor{green}{rgb}{0,0.65,0}


\newcommand{\orangebox}[2]{
\setbeamercolor{uppercol}{fg=white,bg=pacificorange}
\setbeamercolor{lowercol}{fg=black,bg=pacificcream}

\begin{beamerboxesrounded}[upper=uppercol,lower=lowercol,shadow=false]{
#1 }  #2
\end{beamerboxesrounded} }


\newcommand{\greenbox}[2]{
\setbeamercolor{uppergrn}{fg=white,bg=pacificgreen}
\setbeamercolor{lowercol}{fg=black,bg=pacificcream}

\begin{beamerboxesrounded}[upper=uppergrn,lower=lowercol,shadow=false]{
#1 }  #2
\end{beamerboxesrounded} }

\newcommand{\bluebox}[2]{
\setbeamercolor{upperblue}{fg=white,bg=pacificblue}
\setbeamercolor{lowercol}{fg=black,bg=pacificcream}

\begin{beamerboxesrounded}[upper=upperblue,lower=lowercol,shadow=false]{
#1 }  #2
\end{beamerboxesrounded} }

\newcommand{\graybox}[2]{
\setbeamercolor{uppergray}{fg=white,bg=pacificgray}
\setbeamercolor{lowercol}{fg=black,bg=pacificcream}

\begin{beamerboxesrounded}[upper=uppergray,lower=lowercol,shadow=false]{
#1 }  #2
\end{beamerboxesrounded} }

\newcommand{\grassgreenbox}[2]{
\setbeamercolor{uppergrass}{fg=white,bg=grassgreen}
\setbeamercolor{lowercol}{fg=black,bg=pacificcream}

\begin{beamerboxesrounded}[upper=uppergrass,lower=lowercol,shadow=false]{
#1 }  #2
\end{beamerboxesrounded} }

\newcommand{\boxgreen}[1]{
  \tikz[na]{\node[fill=green!20,anchor=base] (n2)
    {#1};
  }
}

\newcommand{\boxred}[1]{
  \tikz[na]{\node[fill=red!20,anchor=base] (n2)
    {#1};
  }
}

\newcommand{\boxblue}[1]{
  \tikz[na]{\node[fill=blue!20,anchor=base] (n1)
    {#1};
  }
}

\setbeamertemplate{navigation symbols}{}
\setbeamertemplate{footline}{}
\newenvironment{orangeitemize}{\setbeamertemplate{itemize item}{\textcolor{orange}{$\bullet$}}\begin{itemize}}{\end{itemize}}
\newenvironment{grassgreenitemize}{\setbeamertemplate{itemize item}{\textcolor{grassgreen}{$\bullet$}} \begin{itemize}}{\end{itemize}}
\newenvironment{greenitemize}{\setbeamertemplate{itemize item}{\textcolor{green}{$\bullet$}} \begin{itemize}}{\end{itemize}}
\newenvironment{blackitemize}{\setbeamertemplate{itemize item}{\textcolor{black}{$\bullet$}} \begin{itemize}}{\end{itemize}}

%%------------ End of different colors for boxes----------------------

%%---------------Some color command-----------------------------------
\newcommand{\mygreen}[1]{{\color[rgb]{0,.65,0} #1}}
\newcommand{\mywhite}[1]{{\color[rgb]{1.0,1.0,1.0} #1}}
\newcommand{\myred}[1]{{\color[rgb]{0.65,0.0,0.0} #1}}
\newcommand{\myblue}[1]{{\color[rgb]{0.0,0.0,1.0} #1}}
\newcommand{\myblack}[1]{{\color[rgb]{0.0,0.0,0.0} #1}}
\newcommand{\mygray}[1]{{\color[rgb]{.15,.35,.60} #1}}
\newcommand{\mythemec}[1]{{\color[RGB]{51,108,121} #1}}
\newcommand{\mydarkred}[1]{{\color[rgb]{.6,.1,.1} #1}}
\newcommand{\mydarkblue}[1]{{\color[rgb]{.1,.1,.9} #1}}
\newcommand{\mygreenback}[1]{{\color[rgb]{.75,1,.75} #1}}
\newcommand{\myorange}[1]{{\color[rgb]{.76,.39,.13} #1}}
\newcommand{\mygrass}[1]{{\color[rgb]{.19,.64,.13} #1}}
\newcommand{\mysierp}[1]{{\color[RGB]{209,28,209} #1}}
\definecolor{cornflowerblue}{RGB}{71,75,182}
\definecolor{somegreen}{RGB}{51,108,121}
\definecolor{moregray}{RGB}{82,81,81}

\newcommand{\textb}[1]{\textcolor{cornflowerblue}{\textbf{#1}}}
\newcommand{\texto}[1]{\textcolor{orange}{\textbf{#1}}}

\newcommand{\tcb}[1]{\textcolor{blue}{{#1}}}
\newcommand{\tco}[1]{\textcolor{deepviolet}{{#1}}}
\newcommand{\tcm}[1]{\mygrass{{#1}}}
\newcommand{\tcg}[1]{\textcolor{pacificgreen}{{#1}}}
\newcommand{\CC}{\tenfour{\textcolor{red}{C}}}
\newcommand{\tCC}{\tenfour{\textcolor{cyan}{\tilde{C}}}}

\newcommand{\smallhead}[1]{\mygrass{#1}}

%****************** Title slide *******************************************
\usepackage[normalem]{ulem}
\newcommand{\tanbui}[2]{\textcolor{blue}{\sout{#1}} \textcolor{red}{#2}}
\newcommand{\GM}[2]{\mc{N}\left( #1, #2 \right)}
\newcommand{\yobs}{\mythemec{\bs{y}^{\text{obs}}}}
\newcommand{\DD}[2] {\ensuremath{\frac{d {#1}}{d {#2}}}}
\renewcommand{\u}{\myblue{\mb{u}}}
\newcommand{\equaldef}{\stackrel{\mathrm{def}}{=}}
\newcommand{\Y}{Y}
\newcommand{\C}{\mc{C}}

\newcommand{\dt}{\Delta t}
\newcommand{\Ib}{ \textbf{I}}
\newcommand{\Gb}{ \textbf{G}}
\newcommand{\MSE}[1]{ \nor{#1}_2^2}
\newcommand{\MSEsqrt}[1]{ \nor{#1}_2}
\newcommand{\NN}[1]{ \Psi \LRp{#1}}

\newcommand{\ui}[1]{ {\ub}^{{#1}}}
\newcommand{\ut}[1]{ \tilde{\ub}^{{#1}}}
\newcommand{\ubar}[1]{ \bar{\ub}^{{#1}}}

% \newcommand{\vb}{\mb{v}}
\newcommand{\vi}[1]{ {\vb}^{{#1}}}
\newcommand{\vt}[1]{ \tilde{\vb}^{{#1}}}
\newcommand{\vbar}[1]{ \bar{\vb}^{{#1}}}

\newcommand{\er}[1]{ {\bs{e}}_{\text{ML}}^{{#1}}}
\newcommand{\err}[1]{ {\varepsilon}_{\text{ML}}^{{#1}}}
\newcommand{\ermc}[1]{ {\bs{e}}_{\text{MC}}^{{#1}}}
\newcommand{\ermcr}[1]{ {\varepsilon}_{\text{MC}}^{{#1}}}

\newcommand{\hai}[1]{\textcolor{red}{HAI: #1}}
\newcommand{\haians}[1]{\textcolor{blue}{HAI ANSWER: #1}}

\newcommand{\eval}[2][\right]{\relax \ifx#1\right\relax \left.\fi#2#1\rvert}
\newcommand{\uij}[2]{{\boldsymbol u}_{x_{#1}}^{{#2}}}

\newcommand{\xb}{\bs{x}}
\renewcommand{\d}{d}
\renewcommand{\u}{u}

\newcommand{\eg}[1]{ {\varepsilon}^{{#1}}}
\newcommand{\ev}[1]{ {\bs{e}_{\text{ML}}^{{#1}}}}
% \newcommand{\One}{\mathds{1}}
\renewcommand{\epsilon}{\varepsilon}
\newcommand{\vb}{\bs{v}}
\newcommand{\ubr}{\vb}
\newcommand{\epsb}{\bs{\epsilon}}
\newcommand{\ubh}{\hat{\ub}}
\newcommand{\TNet}{\texttt{TNet}}
\newcommand{\TBNN}{\texttt{TBNN}}

% \newtheorem{theorem}{Theorem}[section]
%\newtheorem{corollary}{Corollary}[theorem]
%\newtheorem{lemma}[theorem]{Lemma}
% \newtheorem{proposition}[theorem]{Proposition}
%\newtheorem{definition}[theorem]{Definition}


%\newcommand{\TNet}{\texttt{TNet}}

\newcommand{\bs}[1]{\boldsymbol{#1}}
\newcommand{\nt}{{n_{\mathrm{t}}}}
\newcommand{\n}{n}
\newcommand{\m}{m}
\newcommand{\pb}{\bs{u}}
\newcommand{\yb}{\bs{y}}
\newcommand{\nor}[1]{\left\| #1 \right\|}
\newcommand{\snor}[1]{\left| #1 \right|}
\newcommand{\DNN}{\Psi}
\newcommand{\bb}{{\bf b}}
\newcommand{\W}{W}
\newcommand{\halfv}[1]{\frac{#1}{2}}
\renewcommand{\P}{U}
\newcommand{\U}{\P}
\newcommand{\G}{\boldsymbol{\mathrm F}}
\newcommand{\B}{B}
\newcommand{\One}{\mathds{1}}
\newcommand{\WO}{\W^\text{}}
\newcommand{\bbO}{\bb^\text{}}
\newcommand{\I}{I}
\newcommand{\ybbar}{\overline{\yb}}
\newcommand{\pbbar}{\overline{\pb}}
\newcommand{\ubbar}{\pbbar}
\newcommand{\ub}{\pb}
\newcommand{\ubNDL}{\ub^\text{nDNN}}
\newcommand{\ybobs}{\bs{y}^{obs}}
\newcommand{\F}{\mc{F}}

%=============================================================
%% \newcommand{\WI}{\W^\text{}}
%% \newcommand{\bbI}{\bb^\text{}}
%% \newcommand{\Pbar}{\overline{\P}}
\newcommand{\Ybar}{\overline{\Y}}
\newcommand{\Ucov}{\Gamma}
\newcommand{\Ucovinv}{\Ucov^{-1}}
\newcommand{\Ycov}{\Lambda}
\newcommand{\Ycovinv}{\Lambda^{-1}}
\newcommand{\ubTNET}{\ub^{\texttt{TNet}}}
\newcommand{\ubMCDL}{\ub^{\texttt{mcDNN}}}
\newcommand{\mcDNN}{\texttt{mcDNN}}
\newcommand{\Pbar}{\overline{\P}}
\newcommand{\nDNN}{\texttt{nDNN}}

\newcommand{\ms}[1]{\mathcal{#1}} %mathcal
\newcommand{\Range}{{\mc{R}}}
\newcommand{\N}{\mathcal{N}} 
\newcommand{\A}{{\ms{A}}}

\def\edgedist{2.5}

\tikzstyle{format} = [draw, thin, fill=blue!20]
\tikzstyle{medium} = [ellipse, draw, thin, fill=green!20]
\tikzstyle{output} = [ellipse, draw, thin, fill=red!20]

\title{
  {Title of your talk: PHO-Group at UT}
  }

\author{Author 1, Author 2, PHO-Group at UT}

\institute[]{%
  \small
  \mythemec{Name of group: PHO-Group at UT} \\
  % \vspace{1em}
   \includegraphics[width=0.40\textwidth]{Figs/logos/logo.pdf} \hspace{2em}
   \includegraphics[width=0.3\textwidth]{Figs/logos/ase-logo-formal.pdf}\\ 
  \mysierp{Our research are funded by ..., PHO-Group at UT}
% {Department of Aerospace Engineering and Engineering Mechanics} \\
% {The Oden Institute for Computational Engineering and Sciences}\\
% \myorange{The University of Texas at Austin}
%
%\vspace{0.45cm}
%Joint work with Carsten Burstedde, Omar Ghattas, James R. Martin, Georg Stadler, and Lucas Wilcox \\
%% \pgfuseimage {utbig} \\
%%  \pgfuseimage {ase} $\quad $ \pgfuseimage {icesbig} 
}

\date[Learn2Solve]{%\small\myorange{03/25/2024}
%\Large{This research is partially supported by NSF and DOE}
}
